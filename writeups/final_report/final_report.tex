
\documentclass[10pt]{article}

\usepackage{pgfplots}
\usepackage{tikz}
\usepackage{amsmath}
\usepackage{amssymb}     % for mathbb
\usepackage{enumerate}
\usepackage{xfrac}
\usepackage{float}
\usepackage{hyperref}
\usepackage{algorithmic}
\usepackage{algorithm}
\usepackage{booktabs}    % for beautiful tables
\usepackage{siunitx}     % for si units
\usepackage{color}
\usepackage{titlesec}    % for header spacing
\usepackage{caption}     % for caption*
\usepackage[capitalize]{cleveref}

\usepackage[backend=bibtex,style=ieee]{biblatex}

% set margins
\addtolength{\oddsidemargin}{-.875in}
\addtolength{\evensidemargin}{-.875in}
\addtolength{\textwidth}{1.75in}
\addtolength{\topmargin}{-.875in}
\addtolength{\textheight}{1.75in}

\setlength{\parindent}{0cm}        % no paragraph indentations
\setlength{\parskip}{0.5em}        % small paragraph spacing

\titlespacing\section{0pt}{10pt plus 4pt minus 2pt}{0pt plus 2pt minus 2pt}
\titlespacing\subsection{0pt}{10pt plus 4pt minus 2pt}{0pt plus 2pt minus 2pt}
\titlespacing\subsubsection{0pt}{10pt plus 4pt minus 2pt}{0pt plus 2pt minus 2pt}

\DeclareMathOperator*{\argmax}{arg\,max}

\newcommand{\calcium}[0]{Ca\textsuperscript{2+}}
\newcommand{\todo}[1]{\textcolor{red}{#1}}
\providecommand{\tw}[1]{{\tw[TIM: #1]}}

\AtEveryBibitem{
\ifentrytype{inproceedings}{
    \clearlist{address}
    \clearlist{publisher}
    \clearname{editor}
    \clearlist{organization}
    \clearfield{url}  
    \clearfield{doi}  
    \clearfield{pages}  
    \clearlist{location}
 }{}
 }

\addbibresource{ref.bib}
\renewcommand*{\bibfont}{\footnotesize}

\begin{document}

\begin{center}
    {\LARGE Clustering of Neurons from Large-Scale Calcium Imaging Data}

    Stats 306b Project - Final Report

    Mohammad Ebrahimi and Tim Wheeler

    Spring 2015
\end{center}

\begin{abstract}
Fluorescent imaging allows for the analysis of signalling behavior on a per-neuron basis in anesthetized and awake behaving animals. % ~\cite{Mukamel2009}
Existing methods monitor {\calcium} dynamics over a large region, but recently developed automated methods allow for isolating signals and associating them with particular neurons.
The resulting set of candidate neurons suffers from the presence of background structures such as blood vessels.
This project applies clustering methods to candidate objects in a one-photon {\calcium} dataset for the classification of neurons from background structures.
\todo{Hierarchical clustering was performed to identify preliminary classes.}
\todo{Principle component analysis was performed to identify preliminary important features and a clustering analysis was conducted using the first two principle components.}
\todo{Sparse clustering and sparse principle component analysis were conducted to identify a reduced featureset which captures the majority of the variation in the data.}
\todo{Group validation was performed by comparing the clusters obtained in the first dataset to the clusters obtained in a withheld dataset.}
\todo{Results indicate...}
\end{abstract}


\section{Introduction}

\todo{Introduce away}

% One proposed method leverages spatio-temporal sparseness in elevated [\calcium] levels to identify neuronal spikes and associate them with particular regions in the source frames.
% The resulting work suffers from the presence of background structure such as blood vessels.
% This project aims to apply unsupervised learning methods to identify classes of identified objects from one-photon {\calcium} imaging in the primary visual cortex (V1) of awake behaving mice to aid in the automatic unsupervised classification of neurons from background structures.

\todo{Cover prior work}

\todo{Paper organization and results}


\section{Data Source}

Previous work by \citeauthor{Mukamel2009} obtained one-photon {\calcium} imaging recordings from the primary visual cortex of awake behaving mice.
Their work isolated candidate objects from cortex recordings of mice performing a go/no-go task for \num{30}-minute sessions over three consecutive days.
This work uses the isolated candidate objects, including frame images assocated with each of the extracted independent components and averaged signal traces associated with each candidate.

The complete cortex recording was segmented into a four-by-four grid.
Candidate object images are \num{500} $\times$ \num{500} pixel black and white frame averages that show the strength of their respective independent component over its corresponding segmented field of view. 
The pixel values correspond to the flourescent signal of the identified object, which may be a neuron, part of a blood vessel, or background noise.

Independent component traces were sampled at \num{6.7}\si{Hz}, totalling \num{1200} samples over each \num{30}-minute recording. The component traces of neurons possess indicative spikes with tapered decays corresponding to neuron activity, but also contain significant background noise.

Each recording contains a few thousand extracted objects, the idendity and quantity of which will vary between recordings.
A previously conducted analysis of the dataset found that it contains approximately 50\% actual neurons~\todo{CITE?}.

\section{Features}

A set of \num{42} candidate features was extracted from the segmented image data and associated intracellular [\calcium] signals as is listed in~\cref{table:allfeatures}.
The set includes \emph{image features} obtained from the grayscale frame averages and \emph{trace features} from the independent component time series traces.

\begin{table}[h]
  \centering
  \small
  \caption{Candidate fatures extracted for each independent component in the {\calcium} imaging dataset}
  \label{table:allfeatures}
  \begin{tabular}{lll}
    \toprule
    \multicolumn{3}{c}{Image Features} \\
    \midrule

    area & pixels & peak region area  \\
    \addlinespace[2pt]
    perimeter & pixels & peak region perimeter \\
    \addlinespace[2pt]
    skew & - & peak region skew \\
    \addlinespace[2pt]
    peak & intensity &  maximum component intensity value  \\
    \addlinespace[2pt]
    roundness & - &  ratio proportional to area over squared perimeter\\
    \addlinespace[2pt]
    contrast & - &  contrast of normalized pixels in the component subregion \\
    \addlinespace[2pt]
    peak region mean & intensity &  mean intensity from pixels in peak region \\
    \addlinespace[2pt]
    peak region stdev & intensity\textsuperscript{2} & standard deviation of intensity from pixels in peak region \\
    \addlinespace[2pt]
    subregion mean & intensity &  mean intensity from pixels in component subregion \\
    \addlinespace[2pt]
    subregion stdev & intensity\textsuperscript{2} &  standard deviation of intensity from pixels in component subregion \\
    \addlinespace[2pt]
    primary directionality & - &  the directionality histogram bin of greatest value \\
    \addlinespace[2pt]
    directionality entropy & Hart &  the base-\num{10} entropy for the directionality histogram \\
    \midrule
    \multicolumn{3}{c}{Trace Features} \\
    \midrule

    \todo{trace 1} & \todo{???} &  \todo{a trace feature} \\
    \addlinespace[2pt]
    \todo{trace 2} & \todo{???} &  \todo{a trace feature} \\
    \addlinespace[2pt]
    \todo{trace 3} & \todo{???} &  \todo{a trace feature} \\
    \addlinespace[2pt]
    \todo{trace 4} & \todo{???} &  \todo{a trace feature} \\
    \addlinespace[2pt]
    \todo{trace 5} & \todo{???} &  \todo{a trace feature} \\
    \addlinespace[2pt]
    \todo{trace 6} & \todo{???} &  \todo{a trace feature} \\
    \addlinespace[2pt]
    \todo{trace 7} & \todo{???} &  \todo{a trace feature} \\
    \addlinespace[2pt]
    \todo{trace 8} & \todo{???} &  \todo{a trace feature} \\
    \addlinespace[2pt]
    \todo{trace 9} & \todo{???} &  \todo{a trace feature} \\
    \addlinespace[2pt]
    \todo{trace 10} & \todo{???} &  \todo{a trace feature} \\
    \addlinespace[2pt]
    \todo{trace 11} & \todo{???} &  \todo{a trace feature} \\
    \addlinespace[2pt]
    \todo{trace 12} & \todo{???} &  \todo{a trace feature} \\
    \addlinespace[2pt]
    \todo{trace 13} & \todo{???} &  \todo{a trace feature} \\
    \addlinespace[2pt]
    \todo{trace 14} & \todo{???} &  \todo{a trace feature} \\
    \addlinespace[2pt]
    \todo{trace 15} & \todo{???} &  \todo{a trace feature} \\
    \addlinespace[2pt]
    \todo{trace 16} & \todo{???} &  \todo{a trace feature} \\
    \addlinespace[2pt]
    \todo{trace 17} & \todo{???} &  \todo{a trace feature} \\
    \addlinespace[2pt]
    \todo{trace 18} & \todo{???} &  \todo{a trace feature} \\
    \addlinespace[2pt]
    \todo{trace 19} & \todo{???} &  \todo{a trace feature} \\
    \addlinespace[2pt]
    \todo{trace 20} & \todo{???} &  \todo{a trace feature} \\
    \addlinespace[2pt]
    \todo{trace 21} & \todo{???} &  \todo{a trace feature} \\
    \addlinespace[2pt]
    \todo{trace 22} & \todo{???} &  \todo{a trace feature} \\
    \addlinespace[2pt]
    \todo{trace 23} & \todo{???} &  \todo{a trace feature} \\
    \addlinespace[2pt]
    \todo{trace 24} & \todo{???} &  \todo{a trace feature} \\
    \addlinespace[2pt]
    \todo{trace 25} & \todo{???} &  \todo{a trace feature} \\
    \addlinespace[2pt]
    \todo{trace 26} & \todo{???} &  \todo{a trace feature} \\
    \addlinespace[2pt]
    \todo{trace 27} & \todo{???} &  \todo{a trace feature} \\
    \addlinespace[2pt]
    \todo{trace 28} & \todo{???} &  \todo{a trace feature} \\
    \addlinespace[2pt]
    \todo{trace 29} & \todo{???} &  \todo{a trace feature} \\
    \addlinespace[2pt]
    \todo{trace 30} & \todo{???} &  \todo{a trace feature} \\

    \bottomrule
  \end{tabular}
\end{table}

\subsection{Image Features}

\begin{figure}[h]
    \centering
    \begin{minipage}{.33\textwidth}
      \centering
      \includegraphics[width=.9\linewidth]{frame_2.png}
      \caption*{\footnotesize A) Clean neuron source frame.}
      \label{fig:frame1}
    \end{minipage}%
    \begin{minipage}{.33\textwidth}
      \centering
      \includegraphics[width=.9\linewidth]{frame_3.png}
      \caption*{\footnotesize B) Background structure. }
      \label{fig:frame2}
    \end{minipage}
    \begin{minipage}{.33\textwidth}
      \centering
      \includegraphics[width=.9\linewidth]{frame_4.png}
      \caption*{\footnotesize C) Large neuron. }
      \label{fig:frame3}
    \end{minipage}
    \caption{\footnotesize Examples of frame data. (A) contains a clean neuron source frame. (B) contains background structure in the form of blood vessels. The FWHM peak region is colored magenta. (C) contains an unknown structure which is likely a neuron. Here the peak region is much larger. }
\end{figure}

A set of features were extracted for each image frame.
A frame is a \num{500} by \num{500} pixel monochromatic matrix containing the average IC over the segemented field of view. 
Image frame values vary between background noise on the order of \num{+-3} but contain sharp peak values as high as \num{100}.
A total of eleven scalar features and sixteen histogrammed directionality features were extracted from each image.

Peak blobs are identified using the full-width half-maximum (FWHM). 
All pixels above half the maximum value are considered.
A simple flood-fill algoithm is used to find the largest connected blob, hereafter referred to as the peak region.

Several features are derived directly from the peak region.
The \emph{peak region size} is the number of pixels in the peak region, and corresponds roughly to the area of the peak region.
The \emph{peak region perimeter} is the number of pixels on the exterior of the peak region, and corresponds roughly to the actual perimeter of the peak region. 
An approximate measure of \emph{roundness} can be obtained by computing the ratio of the peak region size and perimeter:

$$
\text{roundness} = 4\pi \frac{\text{area}}{\text{perimeter}^2}
$$

\noindent
where a roundness of one corresponds roughly to a perfect circle.
The peak region \emph{skew} is obtained by computing the $xy$-correlation between the square subsection of the image of \num{21} pixel side length centered at the peak region center.

The \emph{peak} value of the peak region was extracted, as well as the peak region \emph{mean} and \emph{standard deviation}.

A subimage of \num{24} pixel side length was extracted about the peak region center.
The \emph{entropy} was computed over the normalized pixel values.
The mean and standard deviation of the subimage was also computed.

A measure of \emph{coarseness} and \emph{directionality} was obtained using the Tamura definition~\cite{Tamura1978}.
Contrast is a measure of image sharpness. 
The contrast was calculated using

$$
F_\text{con} = \frac{\sigma}{\alpha_4^z} \quad \text{with} \quad \alpha_4 = \frac{\mu_4}{\sigma^4}
$$

\noindent
where $\mu_4 = \frac{1}{nm} \sum_{i=1}^n \sum_{j=1}^m (IC(i,j)-\mu)^4$
is the fourth moment about the mean $\mu$, $z$ is \num{0.25} from experiment, and $\sigma^2$ is the variance of the image values.

Directionality was computed by obtaining the approximate horizontal and vertical derivatives from convolution with the following $3\times3$ matrices, respectively $\Delta_V$ and $\Delta_H$:

$$
\begin{bmatrix}
-1 & 0 & 1 \\ -1 & 0 & 1 \\ -1 & 0 & 1
\end{bmatrix} \qquad
\begin{bmatrix}
-1 & -1 & -1 \\
0 & 0 & 0 \\
1 & 1 & 1
\end{bmatrix}
$$

\noindent
and then computing $\theta = \frac{\pi}{2} + \tan^{-1}\frac{\Delta_V(i,j)}{\Delta_H(i,j)}$ for each pixel in the image.
\todo{These values are discretized into a sixteen bin histogram which is used as a feature.}

\subsection{Trace Features}

Calcium activity for a perfectly recorded neuron is in the form of temporally sparse spikes that start with a fast jump followed by exponential decay. The sparsity of spikes and decay time constant variy between neurons~\cite{Mukamel2009}. 
An example trace is shown in \cref{fig:trace}.

The distribution of each neuronal trace is estimated in order to extract the relevant features. For each trace a \emph{high threshold} and \emph{low threshold} are calculated where the \emph{CDF} used is \num{0.97} and \num{0.03} respectively. In addition, every time the trace crosses the high threshold and low threshold line, a high crossing and low crossing event is detected. These events can be described by a \emph{rise time}, \emph{fall time} and \emph{pick value}. Average and variance of these three variables are also computed. The \emph{ratio of mean pick value to threshold value} of high or low crossings are also extraced.

The sparsity of neural activity is also known to be a good indicator. Average and variance of \emph{crossing event intervals} for high and low crossings are taken as one of the sparsity features where each interval is the time distance between two picks of the crossing events.  The ratio of \emph{total crossing duration} to the total time is calculated for each family separately and taken as a sparsity measure. Neurons are expected to have primarily high crossing events. The \emph{number of high crossing events}, \emph{number of low crossing events} and the ratio of these two are also included. The final trace feature vector includes \num{30} indices.

% \begin{figure}[h]
%     \centering

%     \begin{minipage}{1\textwidth}
%       \centering
%       \includegraphics[width=0.5\linewidth]{trace.pdf}
%       \caption{\footnotesize Example neuron Calcium trace. \textcolor{blue}{blue *} indicates start, pick and end of high crossing events and \textcolor{red}{red *} indicates the same time points in low crossing events }
%       \label{fig:trace}
%     \end{minipage}
  
% \end{figure}


\section{Anaylsis}

\todo{Intro discussion here.}

\subsection{Full Hierarchical Clustering}

\todo{Wherein we just perform straight up agglomerative hierarchical clustering to see what we get.}

\todo{Include scatter plots with important features, etc.}

\subsection{Spectral Hierarchical Clustering}

\todo{Wherein we use the first two principle components to perform clustering}

\subsection{Sparse Hierarchical Clustering}

\todo{Wherein we use sparcl to perform sparse clustering}

\subsection{Group Validation}

\todo{Wherein we compare the clusters obtained in dataset A to those found in dataset B}

\section{Conclusions and Future Work}

\todo{Recap what was done}

\todo{Discuss the primary results}

\todo{Some ideas for future work.}

% \newpage

\printbibliography

\end{document}