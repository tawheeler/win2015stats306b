
\documentclass[11pt]{article}

\usepackage{pgfplots}
\usepackage{tikz}
\usepackage{amsmath}
\usepackage{amssymb}     % for mathbb
\usepackage{enumerate}
\usepackage{xfrac}
\usepackage{float}
\usepackage{hyperref}
\usepackage{algorithmic}
\usepackage{algorithm}
\usepackage{booktabs}    % for beautiful tables
\usepackage{siunitx}     % for si units
\usepackage{color}
\usepackage{titlesec}    % for header spacing
\usepackage[capitalize]{cleveref}

\usepackage[backend=bibtex,style=ieee]{biblatex}

% set margins
\addtolength{\oddsidemargin}{-.875in}
\addtolength{\evensidemargin}{-.875in}
\addtolength{\textwidth}{1.75in}
\addtolength{\topmargin}{-.875in}
\addtolength{\textheight}{1.75in}

\setlength{\parindent}{0cm}        % no paragraph indentations
\setlength{\parskip}{0.5em}        % small paragraph spacing

\titlespacing\section{0pt}{10pt plus 4pt minus 2pt}{0pt plus 2pt minus 2pt}
\titlespacing\subsection{0pt}{10pt plus 4pt minus 2pt}{0pt plus 2pt minus 2pt}
\titlespacing\subsubsection{0pt}{10pt plus 4pt minus 2pt}{0pt plus 2pt minus 2pt}

\DeclareMathOperator*{\argmax}{arg\,max}

\newcommand{\calcium}[0]{Ca\textsuperscript{2+}}
\newcommand{\todo}[1]{\textcolor{red}{#1}}
\providecommand{\tw}[1]{{\tw[TIM: #1]}}

\AtEveryBibitem{
\ifentrytype{inproceedings}{
    \clearlist{address}
    \clearlist{publisher}
    \clearname{editor}
    \clearlist{organization}
    \clearfield{url}  
    \clearfield{doi}  
    \clearfield{pages}  
    \clearlist{location}
 }{}
 }

\addbibresource{ref.bib}
\renewcommand*{\bibfont}{\footnotesize}

\begin{document}

\begin{center}
    {\LARGE Clustering of Neurons from Large-Scale Calcium Imaging Data}

    Stats 306b Project - Progress Report

    Mohammad Ebrahimi and Tim Wheeler

    Spring 2015
\end{center}

\section{Introduction}

Fluorescent imaging allows for the analysis of signalling behavior on a per-neuron basis in anesthetized and awake behaving animals~\cite{Mukamel2009}. 
Existing methods monitor {\calcium} dynamics over a large region, but recently developed automated methods allow for isolating signals and associating them with particular neurons.
One proposed method leverages spatio-temporal sparseness in elevated [\calcium] levels to identify neuronal spikes and associate them with particular regions in the source frames.
The resulting work suffers from the presence of background structure such as blood vessels.
This project aims to apply unsupervised learning methods to identify classes of identified objects from one-photon {\calcium} imaging in the primary visual cortex (V1) of awake behaving mice to aid in the automatic unsupervised classification of neurons from background structures.

\section{Dataset}

This project will use an existing dataset obtained using one-photon {\calcium} imaging in the primary visual cortex of awake behaving mice performing a \num{30}-minutes go/nogo task.
Video recordings are available for three consecutive days of \num{30}-minute sessions each.
Data are presented as frame images associated with each of the extracted independent components (ICs).
Images are black and white frame averages that show the strength of each IC over the segmented field of view. 
Each IC indicates the flourscent signal of one independent object which may be a neuron, part of a blood vessel, or background noise.

The ICs are sampled at \num{6.7}\si{Hz}, totalling \num{1200} samples over each \num{30}-minute recording. 
Each recording contains a few thousand extracted objects, the idendity and quantity of which will vary between recordings.
The original raw videos are also available and could provide additional information.

\section{Features}

Offline clustering methods will be used to identify relevant object classes from segmented image data and associated intracellular [\calcium] signals.
The unsupervised learning methods employed in this work require a set of features that capture the differences between neurons and background structure. 
Features were extracted from both the black and white image frame averages and the independent component time series traces.

\subsection{Image Features}

\begin{figure}[h]
    \centering
    \begin{minipage}{.33\textwidth}
      \centering
      \includegraphics[width=.9\linewidth]{frame_2.png}
      \caption{\footnotesize Clean neuron source frame.}
      \label{fig:test1}
    \end{minipage}%
    \begin{minipage}{.33\textwidth}
      \centering
      \includegraphics[width=.9\linewidth]{frame_3.png}
      \caption{\footnotesize Background structure. }
      \label{fig:test2}
    \end{minipage}
    \begin{minipage}{.33\textwidth}
      \centering
      \includegraphics[width=.9\linewidth]{frame_4.png}
      \caption{\footnotesize Large neuron. }
      \label{fig:test2}
    \end{minipage}
    \caption{Examples of frame data. (1) contains a clean neuron source frame. (2) contains background structure in the form of blood vessels. The FWHM peak region is colored magenta. (3) contains an unknown structure which is likely a neuron. Here the peak region is much larger. }
\end{figure}

A set of features were extracted for each image frame.
A frame is a \num{500} by \num{500} pixel monochromatic matrix containing the average IC over the segemented field of view. 
Image frame values vary between background noise on the order of \num{+-3} but contain sharp peak values as high as \num{100}.
A total of eleven scalar features and sixteen histogrammed directionality features were extracted from each image.

Peak blobs are identified using the full-width half-maximum (FWHM). 
All pixels above half the maximum value are considered.
A simple flood-fill algoithm is used to find the largest connected blob, hereafter referred to as the peak region.

Several features are derived directly from the peak region.
The \emph{peak region size} is the number of pixels in the peak region, and corresponds roughly to the area of the peak region.
The \emph{peak region perimeter} is the number of pixels on the exterior of the peak region, and corresponds roughly to the actual perimeter of the peak region. 
An approximate measure of \emph{roundness} can be obtained by computing the ratio of the peak region size and perimeter:

$$
\text{roundness} = 4\pi \frac{\text{area}}{\text{perimeter}^2}
$$

\noindent
where a roundness of one corresponds roughly to a perfect circle.
The peak region \emph{skew} is obtained by computing the $xy$-correlation between the square subsection of the image of \num{21} pixel side length centered at the peak region center.

The \emph{peak} value of the peak region was extracted, as well as the peak region \emph{mean} and \emph{standard deviation}.

A subimage of \num{24} pixel side length was extracted about the peak region center.
The \emph{entropy} was computed over the normalized pixel values.
The mean and standard deviation of the subimage was also computed.

A measure of \emph{coarseness} and \emph{directionality} was obtained using the Tamura definition~\cite{Tamura1978}.
Contrast is a measure of image sharpness. 
The contast was calculated using

$$
F_\text{con} = \frac{\sigma}{\alpha_4^z} \quad \text{with} \quad \alpha_4 = \frac{\mu_4}{\sigma^4}
$$

\noindent
where $\mu_4 = \frac{1}{nm} \sum_{i=1}^n \sum_{j=1}^m (IC(i,j)-\mu)^4$
is the fourth moment about the mean $\mu$, $z$ is \num{0.25} from experiment, and $\sigma^2$ is the variance of the image values.

Directionality was computed by obtaining the approximate horizontal and vertical derivatives from convolution with the following $3\times3$ matrices, respectively $\Delta_V$ and $\Delta_H$:

$$
\begin{bmatrix}
-1 & 0 & 1 \\ -1 & 0 & 1 \\ -1 & 0 & 1
\end{bmatrix} \qquad
\begin{bmatrix}
-1 & -1 & -1 \\
0 & 0 & 0 \\
1 & 1 & 1
\end{bmatrix}
$$

\noindent
and then computing $\theta = \frac{\pi}{2} + \tan^{-1}\frac{\Delta_V(i,j)}{\Delta_H(i,j)}$ for each pixel in the image.
These values are discretized into a sixteen bin histogram which is used as a feature.

\subsection{IC Trace}

\todo{this}

\section{Preliminary Analysis}

A preliminary analysis of the image data set was conducted.
K-means clustering with two clusters was run using roundness, skew, area, perimeter, and peak from the image dataset. 
The data matrix was de-meaned and centered by dividing by the feature-wise standard deviation.
A standard form of K-means was run with \num{100} random initializations.
The results are shown in \cref{fig:components} and \cref{fig:principle}.

\begin{figure}[h]
    \centering
    \begin{minipage}{.75\textwidth}
      \centering
      \includegraphics[width=0.9\linewidth]{kmeans_components3.pdf}
      \caption{\footnotesize Clustering components from K-means.}
      \label{fig:components}
    \end{minipage}%
    \begin{minipage}{.25\textwidth}
      \centering
      \includegraphics[width=0.95\linewidth]{kmeans_svd.pdf}
      \caption{\footnotesize First two principle coordinates. }
      \label{fig:principle}
    \end{minipage}
\end{figure}

Present results do not indicate clear data separation.
Rather, these features appear to exist in a clean continuum.

The addition of the other image features does not improve the dataset separation.
Attempting this results in principal component directions with a near-circular distribution.

Future work will incorporate the IC Trace features and futher improve the image features.
In additional to K-means clustering, heirarchical clustering methods will be used as well.


% \newpage

\printbibliography

\end{document}