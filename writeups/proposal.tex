
\documentclass[11pt]{article}

\usepackage{pgfplots}
\usepackage{tikz}
\usepackage{amsmath}
\usepackage{amssymb}     % for mathbb
\usepackage{enumerate}
\usepackage{xfrac}
\usepackage{float}
\usepackage{hyperref}
\usepackage{algorithmic}
\usepackage{algorithm}
\usepackage{booktabs}    % for beautiful tables
\usepackage{siunitx}     % for si units
\usepackage{color}
\usepackage{titlesec}    % for header spacing

\usepackage[backend=bibtex,style=ieee]{biblatex}

% set margins
\addtolength{\oddsidemargin}{-.875in}
\addtolength{\evensidemargin}{-.875in}
\addtolength{\textwidth}{1.75in}
\addtolength{\topmargin}{-.875in}
\addtolength{\textheight}{1.75in}

\setlength{\parindent}{0cm}        % no paragraph indentations
\setlength{\parskip}{0.5em}        % small paragraph spacing

\titlespacing\section{0pt}{10pt plus 4pt minus 2pt}{0pt plus 2pt minus 2pt}
\titlespacing\subsection{0pt}{10pt plus 4pt minus 2pt}{0pt plus 2pt minus 2pt}
\titlespacing\subsubsection{0pt}{10pt plus 4pt minus 2pt}{0pt plus 2pt minus 2pt}

\DeclareMathOperator*{\argmax}{arg\,max}

\newcommand{\calcium}[0]{Ca\textsuperscript{2+}}
\newcommand{\todo}[1]{\textcolor{red}{#1}}
\providecommand{\tw}[1]{{\tw[TIM: #1]}}

\AtEveryBibitem{
\ifentrytype{inproceedings}{
    \clearlist{address}
    \clearlist{publisher}
    \clearname{editor}
    \clearlist{organization}
    \clearfield{url}  
    \clearfield{doi}  
    \clearfield{pages}  
    \clearlist{location}
 }{}
 }

\addbibresource{ref.bib}
\renewcommand*{\bibfont}{\footnotesize}

\begin{document}

\begin{center}
    {\LARGE Clustering of Neurons from Large-Scale Calcium Imaging Data}

    Stats 306b Project Proposal

    Mohammad Ebrahimi and Tim Wheeler

    Spring 2015
\end{center}

\section{Introduction}

Fluorescent imaging allows for the real-time analysis of neuron behavior in anesthetized and awake behaving animals. 
Existing methods monitor {\calcium} dynamics over a large region, but automated methods for isolating signals and associating them with particular neurons remains in active research.
One proposed method leverages spatio-temporal sparseness in elevated [\calcium] levels to identify neuronal spikes and associate them with particular regions in the source frames.
The resulting work suffers from the presence of background structure such as blood vessels.
This project aims to apply unsupervised learning methods to identify classes of identified objects from two-photon {\calcium} imaging in the cerebellar vermis of awake behaving mice to aid in the automatic classification of neurons from background structure.

\section{Dataset}

This project will use an existing dataset obtained using two-photon {\calcium} imaging in the cerebellar vermis of awake behaving mice~\cite{Mukamel2009}.
The locations of candidate locations and associated intracellular [\calcium] signals are given the form \todo{how is the data presented? Do we have the original raw images?}
\todo{How much data? How many frames?}

\section{Approach}

Offline clustering methods will be used to identify relevant object classes from segmented image data and associated intracellular [\calcium] signals.
The resulting classes may provide crucial insight in distinguishing between the neurons and background structure present in the extracted data.

\begin{table}[h]
    \centering
    \scriptsize
    \begin{tabular}{ccc}
        \toprule
        
        \begin{tabular}{l}
            
            candidate clustering methods \\
            \midrule
            Kernel $k$-means \\
            tree-structured vector quantisation \\
            agglomerative hierarchical clustering \\
            \\
        \end{tabular} & 

        \begin{tabular}{l}
            segmented image data over time \\
            \midrule
            principle components \\
            auto-encoded feature space \\
            \todo{other ideas?} \\
            \\
        \end{tabular} & 

        \begin{tabular}{l}
            \calcium time histories \\
            \midrule
            frequency content \\
            mean peak amplitude \\
            mean peak width \\
            \todo{other ideas?} \\
        \end{tabular} \\

        \bottomrule
    \end{tabular}
\end{table}

\section{Measurement of Success}

Success of the project will be measured in the ability of a clustering method to correctly distinguish between neurons and background structure and identification of the key features used to make the distinction.
A subset of the training data is labelled, and can thus be used to compute a validation score.
Additional success metrics include the identification of subsets of neuron classes.

% \newpage

\printbibliography

\end{document}